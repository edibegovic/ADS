\documentclass[a4paper]{article}
\usepackage[utf8x]{inputenc}
\usepackage[danish]{babel}
\usepackage{utopia}
\usepackage{graphicx}
\usepackage{listings}
\usepackage{cmap}

\title{Catenable Queues - ADS}
\author{Edi Begovic | Høgni Jacobsen | Gergö Koncz}
\date{\today}

\def\arraystretch{1.5}
\def\headline#1{\hbox to \hsize{\hrulefill\quad\lower.3em\hbox{#1}\quad\hrulefill}}

\begin{document} 
\maketitle

\ \\
\noindent
The pseudo-code will reference the queue on which the function is called as \textit{q}.
\ \\
\section*{Question 1}
\headline{-} \ \\

\noindent
The following implementation removes all occurrences of a given item from the queue by defining a temporary queue
object to which we enqueue every item from the original queue that does not match the item for removal. 
\ \\

\noindent
To proceed from the point of the original queue being empty and the temporary queue having all the elements in the proper order, we considered two options. We could dequeue all the elements from the temporary queue and enqueue them to the original queue. However, since the temporary queue is the perfect queue all we need to do is change the reference of the original queue to the temporary queue thus entrusting the destiny of the original emptied queue to the garbage collector. 

\noindent
\begin{lstlisting}[escapeinside={{*}{*}}]

q.removeall(item):

    temporaryQueue = *\textbf{new}* queue

    *\textbf{while}* !q.isEmpty():

        Current = q.dequeue()
        
        *\textbf{if}* (current != item):
            temporaryQueue.enqueue(current) 
    
    q = temporaryQueue

\end{lstlisting}

\newpage
\section*{Question 2}
\headline{-} \ \\
The following implementation 'concatenates' the queue \textit{p} onto the queue \textit{q} by
dequeuing every item of \textit{p} and enqueuing it onto \textit{q}. The speed of this method depends on the size of the second queue, since all of the elements of that queue are 'touched' in this method.
\ \\

\noindent
At the end it makes sense to change the reference of p to null as it references the emptied queue and cannot be collected by the garbage collector. 
\ \\

\noindent
In later questions we will show that if we can use the underlying benefits of the linked list implementation the method is much quicker.
 \\

\noindent

\begin{lstlisting}[escapeinside={{*}{*}}]
    q.enqueueall(p):
        *\textbf{while}* !p.isEmpty():
            q.enqueue(p.dequeue)
    p = null
\end{lstlisting}

\ \\
\section*{Question 3}
\headline{-} \ \\
This implementation of the FIFO queue is based on linked lists thus we can simply set the 
'next'-pointer of the last node (of \textit{q}) to point to the first node of \textit{p}.
\\

\noindent
We are lucky enough that the implementation of the FIFO queue based on linked lists keeps a reference to both the first and the last elements of the queues, therefore no additional tracking of nodes are needed to realize this assignment.
 \\

\noindent

\begin{lstlisting}[escapeinside={{*}{*}}]
    q.enqueueall(p):
        (q.last).next = p.first
\end{lstlisting}
 
\ \\

\newpage
\section*{Question 4}
\headline{-} \ \\

\noindent
As a starting point for our implementation of the FIFO queue (using resizing arrays) the LIFO stack implementation (Sedgewick p 141) was very useful.
\ \\

\noindent
A FIFO queue based on a resizing array would operate on the same principle (with resize() and isEmpty()) that an array is resized 
to double its size once full or half its size once only a quarter of the array's capacity is used. This is achieved by keeping tracking of 
the number of items in the array, once the previously mentioned conditions are met, a new array 
of the appropriate size is created.
\ \\

\noindent
The push function of the stack is similar to the enqueue in the queue. However, the push function places the element at the end of the array, while enqueue would place the elements at the beginning. Before assigning the 0 index of the array we would use the ReverseIterator (starts at the last element of the array) of the implementation to iterate through the array and increase the index of all the elements. The reverseIterator starts at the end of the array therefore we won't be losing elements. Once all the already existing elements are moved one spot, the new element can be assigned to the 0 index. 
\ \\

\noindent
Performance-wise this method is slow but we came to the conclusion that whether we move the elements in enqueue or it has to be done in dequeue. This way the dequeue method stays quick, only decreases the size then takes the last element of the array. 
\ \\


\begin{lstlisting}[escapeinside={{*}{*}}]
    q.enqueue(item):
    	if(N == a.length):
    		resize(2 a.length)
    	for element in a:
    		increase index with one
    	a[0] = item
    		
\end{lstlisting}

\end{document}