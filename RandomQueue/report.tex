\documentclass{tufte-handout}

\usepackage{amsmath}
\usepackage[utf8]{inputenc}
\usepackage{mathpazo}
\usepackage{booktabs}
\usepackage{microtype}
\usepackage{tikz}
\usepackage{pgfplots}
\pgfplotsset{width=9cm,compat=1.13}

\title{Randomqueue Report}
\author{Edi Begovic - Høgni Jacobsen - Gergő Koncz}

\begin{document}
\maketitle
\thispagestyle{empty}

\section{Implementation of Randomqueue}


Our program \texttt{RandomQueue.py} implements the complete API.
The submission at time [14-02-2019 17:03] passed all tests on CodeJudge.

The items are stored in a list in their sequental order.

The \texttt{enqueue} operation respects this order by enqueuing the new elements at the end of the list. 
This assigment is achieved in constant time.

The \texttt{sample} operation uses a random int to select a random element of the list in constant time. 

The \texttt{dequeue} operation selects a random integer in the range of the list as the index of the element to return. After this, the last element of the list gets assigned to the index of the selected element therefor the restructuring of the list  takes constant time.

In our implementation, the iterator uses a temporarily created list to create a random order of the elements. 
Initialising this list takes linear time, it uses the dequeue to create a random order for the elements. After dequeuing all the elements they have to be enqueued back to the list which also takes linear time. 
After this we loop through the temporarily created list. Since the elements are already in random order we can use a simple for loop, that takes linear time as well.


\subsection{Notes}

Alternative consideration for iterator:
\begin{enumerate}
\item Copy the list (linear time) to temporary list
\item Shuffle it with random.shuffle
\item Yield each element with regular for loop
\end{enumerate}

\end{document}

